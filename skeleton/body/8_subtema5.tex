
\section{Figures \& Tables}

The output for a single-column figure is in Figure~\ref{fig_sim}.  Lorem ipsum dolor sit amet, consectetur adipiscing elit, sed do eiusmod tempor incididunt ut labore et dolore magna aliqua. Lorem ipsum dolor sit amet, consectetur adipiscing elit, sed do eiusmod tempor incididunt ut labore et dolore magna aliqua. Lorem ipsum dolor sit amet, consectetur adipiscing elit, sed do eiusmod tempor incididunt ut labore et dolore magna aliqua. 

Lorem ipsum dolor sit amet, consectetur adipiscing elit, sed do eiusmod tempor incididunt ut labore et dolore magna aliqua. Lorem ipsum dolor sit amet, consectetur adipiscing elit, sed do eiusmod tempor incididunt ut labore et dolore magna aliqua. Lorem ipsum dolor sit amet, consectetur adipiscing elit, sed do eiusmod tempor incididunt ut labore et dolore magna aliqua. 

%See Figure~\ref{fig_wide} for a double-column figure; this is always at the top of a following page.


\begin{figure}[hbt!]
\centering
\includegraphics[width=0.75\linewidth]{images/example-image-16x10.pdf}
\caption{Insert figure caption here}
\label{fig_sim}
\end{figure}


\begin{figure*}
\centering
\includegraphics[width=0.8\linewidth]{images/example-image-16x10.pdf}
\caption{Insert figure caption here}
\label{fig_wide}
\end{figure*}


See example table in Table~\ref{table_example}.

\begin{table}[hbt!]
\begin{threeparttable}
\caption{An Example of a Table}
\label{table_example}
\begin{tabular}{llll}
\toprule
\headrow Column head 1 & Column head 2  & Column head 3 & Column head 4\\
\midrule
One\tnote{a} & Two&three three &four\\ 
\midrule
Three & Four&three three\tnote{b} &four\\
\bottomrule
\end{tabular}
\begin{tablenotes}[hang]
\item[]Table note
\item[a]First note
\item[b]Another table note
\end{tablenotes}
\end{threeparttable}
\end{table}
