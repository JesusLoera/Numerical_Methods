\section{El Método de Romberg}

\vspace{0.5cm}

El método de Romberg (\cite{scherer2010computational}) es un método de integración numérica que permite calcular el valor aproximado de las integrales de la forma

\begin{equation*}
    I = \displaystyle\int_{a}^{b} f(x)dx
\end{equation*}

\noindent donde f(x) es una función escalar de una variable real X.

\vspace{0.5cm}

\noindent \textbf{El algoritmo de Romberg}

\vspace{0.5cm}

El método de Romberg consiste en discretizar el intervalo $\left[a,b\right]$ en una malla de puntos equidistantes de manera que la distancia que separa a dos puntos contiguos está dada por 

\begin{equation*}
    h_n = \frac{b-a}{2^{n-1}}
\end{equation*}

\noindent en donde $n \ge 1 $ y $ m \ge 1$.

\vspace{0.5cm}

Este es un método iterativo y recursivo, donde $R_{n,m}$ representa la aproximación a la integral numérica con un error del orden $O(h_{n}^{2m+1})$. Podemos encontrar el valor de $R_{n,m}$ de la siguiente manera:

\begin{equation*}
    R_{1,1} = \frac{1}{2} \left(b-a\right) (f\left(a\right) + f\left(b\right))
\end{equation*}

\begin{equation*}
    R_{n,1} = \frac{h_n}{2} \left[ f\left( a \right) + f\left(b\right) + 2 \displaystyle\sum_{i=1}^{2^{n-1}} f\left(a+ih_n\right) \right]
\end{equation*}

\begin{equation*}
    R_{i,j} = \frac{ 4^{ j-1 } R_{ i, j-1} - R_{ i-1, j-1} }{ 4^{ j-1} -1}
\end{equation*}

El corazón del método de Romberg es el \emph{método de Extrapolación de Richardson}, un método que toma una secuencia convergente para construir otra secuencia que converge más rápido. Este es ampliamente utilizado en un sin fin de métodos numéricos interativos, pero su más icónica aplicación es justamente en la construcción del algoritmo de Romberg.
