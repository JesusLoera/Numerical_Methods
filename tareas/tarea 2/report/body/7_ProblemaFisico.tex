\section{Resolver la ecuación del Cohete}

\vspace{0.5cm}

\noindent \textbf{Problema.} Encuentra el campo $\vec{E}$ debido a un disco cargado de 
radio $2m$ con una carga total $Q=5 C$ a una distancia de $5m$ del
disco colocado sobre el eje del disco.

\vspace{0.5cm}

\textbf{Solución.}

Conocemos la ley de Coulomb para distribuciones continuas de carga (véase \cite{griffiths2005introduction}).

\begin{equation}
    \vec{E} (\vec{r}) = \frac{1}{4\pi \epsilon_0} 
    \displaystyle\int_{S} \frac{\sigma(\vec{r}) \left( \vec{r} - \vec{r}^{'}  \right) }{\left\lVert \vec{r} - \vec{r}^{'} \right\rVert } dA^{'}
    \label{eqn:ley_coulomb}
\end{equation}

En nuestro caso

\begin{equation*}
    \sigma(\vec{r}^{'}) = \frac{Q}{\pi R^2} 
    \quad \text{,} \quad
    \vec{r} = z_0 \hat{k}
    \quad \text{,} \quad
    \vec{r}^{'} = x \hat{i} + y \hat{j}
\end{equation*}

\begin{equation*}
    \vec{r} - \vec{r}^{'} =
    -x \hat{i} - y \hat{j} + z_{0} \vec{k}
    \quad , \quad
    \left\lVert \vec{r} - \vec{r}^{'} \right\rVert 
    = 
    \left( x^2 + y^2 + z_{0}^{2} \right)^{1/2}
\end{equation*}

Sustituyendo las ecuaciones anteriores en la ecuación número \ref{eqn:ley_coulomb}.

\begin{equation}
    \vec{E} (\vec{r}) = \frac{\sigma}{4\pi \epsilon_0} 
    \displaystyle\int_{S} \frac{ -x \hat{i} - y \hat{j} + z_{0} \vec{k} }{\left( x^2 + y^2 + z_{0}^{2} \right)^{3/2}} dA^{'}
\end{equation}

Como estamos integrando sobre la superficie definida por el interior de la circunferencia de la ecuación $ x^2 + y^2 = 4 $, reescribimos a la integral de la siguiente forma.

\begin{equation}
    \vec{E} (\vec{r}) = \frac{\sigma}{4\pi \epsilon_0} 
    \displaystyle\int_{-2}^{2} 
    \displaystyle\int_{- \sqrt{4 - x^2} }^{\sqrt{4 - x^2}}
    \frac{ -x \hat{i} - y \hat{j} + z_{0} \vec{k} }{\left( x^2 + y^2 + z_{0}^{2} \right)^{3/2}} dy dx
\end{equation}

De esta manera, las componentes rectangulares del campo eléctrico estan dadas por:

\begin{equation}
    \longrightarrow
    E_{x} (\vec{r}) =  
    \displaystyle\int_{-2}^{2} 
    \displaystyle\int_{- \sqrt{4 - x^2} }^{\sqrt{4 - x^2}}
    \frac{\sigma}{4\pi \epsilon_0}
    \frac{ -x }{\left( x^2 + y^2 + z_{0}^{2} \right)^{3/2}} dy dx
\end{equation}

\begin{equation}
    \longrightarrow
    E_{y} (\vec{r}) =  
    \displaystyle\int_{-2}^{2} 
    \displaystyle\int_{- \sqrt{4 - x^2} }^{\sqrt{4 - x^2}}
    \frac{\sigma}{4\pi \epsilon_0}
    \frac{ -y }{\left( x^2 + y^2 + z_{0}^{2} \right)^{3/2}} dy dx
\end{equation}

\begin{equation}
    \longrightarrow
    E_{z} (\vec{r}) =  
    \displaystyle\int_{-2}^{2} 
    \displaystyle\int_{- \sqrt{4 - x^2} }^{\sqrt{4 - x^2}}
    \frac{\sigma}{4\pi \epsilon_0}
    \frac{ z_0 }{\left( x^2 + y^2 + z_{0}^{2} \right)^{3/2}} dy dx
\end{equation}

Calculamos las tres ecuaciones anteriores con el programa escrito en el apéndice titulado "Programa del problema físico en Fortran". 

Vea una comparación de las componentes del campo eléctrico obtenidas en con el programa en Fortran con la solución teórica en la tabla~\ref{table_results}.

\begin{table}[hbt!]
    \begin{threeparttable}
    \begin{tabular}{lll}
    \toprule
    \headrow & Solución numérica $N/C$ & Solución teórica $N/C$ \\
    \midrule
    $E_x$ & -4.57552147 & 0 \\ 
    \midrule
    $E_y$ & -7215.09961 & 0 \\ 
    \midrule
    $E_z$ & $1.60803021 \times 10^9$ & $1.6 \times 10^9$ \\ 
    \bottomrule
    \end{tabular}
    \caption{Se utilizó $h_x = 0.0001$ y $h_y = 0.0001$ para la integración.}
    \label{table_results}
\end{threeparttable}
\end{table}

\textbf{Observación:} La solución teórica del problema, usada para comparar con los resultados numéricos, está dada por la ecuación.

\begin{equation}
    \vec{E} =
    \frac{1}{4\pi \epsilon_0} \frac{2QZ_0}{R^2} 
    \left[
        \frac{1}{z_o} - \frac{1}{\sqrt{R^2-z_o^2}}
    \right]
    \hat{z}
\end{equation}