\noindent Desde la invención del cálculo con Newton y Leibniz, en prácticamente todas las ramas de la física se suele recurrir a el cálculo de integrales para definir magnitudes físicas. Por ejemplo, cuando queremos verificar la normalización de una función de onda

\begin{equation*}
  \braket{\psi | \psi} = \displaystyle\int_{-\infty}^{\infty} \psi^{*} \psi dx
\end{equation*}

El cálculo de dichas integrales se puede volver complicado, por ello solemos recurrir a técnicas de integración numérica para resolver el problema de manera computacional.