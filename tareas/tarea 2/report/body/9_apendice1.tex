\section{Programa de integración doble en Python}

\lstset{
language=Python,
basicstyle=\ttfamily\scriptsize,
morekeywords={self},              % Add keywords here
keywordstyle=\ttfamily\scriptsize\color{deepblue},
emph={MyClass,__init__},          % Custom highlighting
emphstyle=\ttfamily\scriptsize\color{deepred},    % Custom highlighting style
stringstyle=\color{deepgreen},
frame=tb,           
commentstyle=\color{red},              % Any extra options here
showstringspaces=false
}

\begin{lstlisting}
    # Programa elaborado por Jesus Eduardo Loera Casas
    # Elaborado el dia 13/02/23
     
    import numpy as np
    
    
    """
    Parametros de integracion
    """
    
    # Definimos el integrando
    def f(x, y):
        return np.sin(x+y)
    
    # Definimos el limite superior de integracion
    def limit_sup(x):
        return 3 + np.exp(x/5.0)
    
    # Definimos el limite superior de integracion
    def limit_inf(x):
        return np.log(x)
    
    # definimos la region de integracion
    xo = 1.0 ; xf = 3.0
    
    # definimos el tamano de subintervalo de integracion
    dx = 0.0001 ; dy = 0.001
    
    """
    Funciones que definen el metodo de integracion
    """
    
    def trapecio_y(yo, yf, dy, xcte):
        N = int((yf-yo)/dy + 1)
        suma = 0.0
        for i in range(1, N):
            suma = suma + f(xcte, yo+i*dy)
        trapecio_y = (0.5*dy)*(f(xcte,yo) + f(xcte,yf) + 2*suma)
        return trapecio_y
    
    def integracion_doble(xo, xf, dx, dy):
        Nx = int((xf-xo)/dx + 1)
        suma = 0.0
        for i in range(1, Nx):
            yo = limit_inf(xo)
            yf = limit_sup(xf)
            suma = suma + trapecio_y(yo, yf, dy, xo + i*dx)
        yo = limit_inf(xo)
        yf = limit_sup(xf)
        aux = trapecio_y(yo,yf,dy,xo) + trapecio_y(yo,yf,dy,xf)
        integral = (0.5*dx)*(aux + 2*suma)
        return integral
    
    """
    Mandamos a llamar al metodo
    """
    
    integral = integracion_doble(xo, xf, dx, dy)
    print("El valor de la integral es: ", integral)
\end{lstlisting}


 