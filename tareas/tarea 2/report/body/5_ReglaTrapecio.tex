\section{La regla del trapecio}

\vspace{0.5cm}

La regla del trapecio (\cite{nakamura1993applied}) es un método de integración numérica que permite calcular el valor aproximado de las integrales de la forma

\begin{equation*}
    I = \displaystyle\int_{a}^{b} f(x)dx
\end{equation*}

\noindent donde f(x) es una función escalar de una variable real X.

El discretizar la región de integración $ \left[ a,b \right] $ en N subintervalos de igual tamaño $h$ delimitados por los puntos $ \{ x_{0}, x_{1}, x_{2}, ..., x_{N-1}, x_N \} $ donde $x_{i+1} > x_{i}$, $x_0=a$ y $x_N = B$ , se puede aproximar el valor de la integral de la siguiente forma:

\begin{equation}
    \displaystyle\int_{a}^{b} f(x) dx = 
    \frac{h}{2} \left\{ 
    f(a) + f(b) + 2\displaystyle\sum_{i=1}^{N-1} f(x_i) 
    \right\}
\end{equation}

\begin{equation}
    h = \frac{b-a}{N}
\end{equation}

Es decir, podemos escribir

\begin{equation}
    \displaystyle\int_{a}^{b} f(x) dx = 
    \frac{h}{2} \left\{ 
    f(a) + f(b) + 2\displaystyle\sum_{i=1}^{N-1} f(a + i \frac{b-a}{N})
    \right\}
\end{equation}