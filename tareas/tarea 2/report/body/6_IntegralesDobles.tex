\section{Integración doble mediante la regla del trapecio}

Consideremos una integral doble definida I sobre una región rectangular.

\begin{equation}
    I = \displaystyle\int_{a}^{b} \displaystyle\int_{c}^{d} 
    f(x,y) dy dx
\end{equation}

Podemos definir a la función G como sigue

\begin{equation}
    G(x) \equiv \displaystyle\int_{c}^{d} f(x,y) dy
\end{equation}

De esta manera, podemos reescribir la integral I como a continuación.

\begin{equation}
    I = \displaystyle\int_{a}^{b} G(x) dx
\end{equation}

Observe que de esta forma I puede ser calculada con la regla del trapecio ordinaria.

\begin{equation} 
    I =
    \displaystyle\int_{a}^{b} G(x) dx = 
    \frac{h}{2} \left\{ 
    G(a) + G(b) + 2\displaystyle\sum_{i=1}^{N-1} G(a + i \frac{b-a}{N})
    \right\}
\end{equation}

Donde, empleando la definición de G nos queda la siguiente ecuación.

\begin{equation} 
    I =
    \frac{h}{2} \left\{ 
        \displaystyle\int_{c}^{d} f(a,y) dy
        + \displaystyle\int_{c}^{d} f(b,y) dy
        + 2\displaystyle\sum_{i=1}^{N-1} {N}\displaystyle\int_{c}^{d} f(a + i \frac{b-a}{N},y) dy
    \right\}
\end{equation}

Observe que $G(x_i)$ también puede ser cálculado fácilmente por la regla del trapecio.

\begin{equation} 
     G(x_i) = 
    \frac{h}{2} \left\{ 
    f(x_i,c) + f(x_i,d) + 2\displaystyle\sum_{i=1}^{N-1} f(x_i, c + i \frac{d-c}{N})
    \right\}
\end{equation}

Puede ver la implementación de este método numérico en Python en el apéndice "Programa integracion doble en Python" y su implementación en Fortran en el apéndice "Programa integracion doble en Fortran".