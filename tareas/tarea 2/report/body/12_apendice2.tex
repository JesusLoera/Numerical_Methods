\section{Programa en Fortran}

\lstset{language=[90]Fortran,
  basicstyle=\ttfamily\scriptsize,
  keywordstyle=\color{red},
  commentstyle=\color{blue},
  breaklines=true,
  showstringspaces=false,
  morecomment=[l]{!\ }% Comment only with space after !
}

\begin{lstlisting}
        ! Programa elaborado por Jesus Eduardo Loera Casas
        ! Fecha 09/02/23
        
        ! En este programa resolvemos integrales dobles con la regla
        ! del trapecio.
        
                ! definimos el integrandpo 
                real function f(x,y)
                    implicit none
                    real :: x,y
                    f = log(x+2*y)
                    return
                end function
        
                ! programa principal
                program main
        
                    implicit none
                    real :: integral, dx, dy, xo, xf, yo, yf
        
                    ! definimos la region de integracion
                    xo = 1.4 ; xf = 2.0
                    yo = 1.0 ; yf = 1.5
        
                    ! definimos el tamano de subintervalo de integracion
                    dx = 0.0001 ; dy = 0.0001
        
                    call integracion_doble(xo, xf, yo, yf, dx, dy, integral)
        
                    write(*,*) "El valor de la integral es: ", integral
        
                end program main
        
                ! subrutina que calcula la integral doble con regla del trapecio
                subroutine integracion_doble(xo, xf, yo, yf, dx, dy, integral)
                    implicit none
                    real :: xo, xf, yo, yf, dx, dy, integral, suma
                    real :: f
                    real :: trapecio_y, aux
                    integer :: Nx, Ny, i
                    Nx = (xf-xo)/dx + 1
                    Ny = (yf-yo)/dy + 1
                    suma = 0.0
                    do i = 1, Nx-1
                        suma = suma + trapecio_y(yo, yf, dy, xo + i*dx)
                    end do
                    aux = trapecio_y(yo,yf,dy,xo) + trapecio_y(yo,yf,dy,xf)
                    integral = (0.5*dx)*(aux + 2*suma)
                end subroutine
        
                ! regla del trapecio para integrar funciones f(cte,y)
                real function trapecio_y(yo, yf, dy, xcte)
                    implicit none
                    real :: yo, yf, dy, suma
                    ! funcion del integrando
                    real :: f, xcte
                    integer :: N, i
        
                    N = (yf-yo)/dy + 1
                    suma = 0.0
                    do i = 1, N-1
                        suma = suma + f(xcte, yo+i*dy)
                    end do
                    trapecio_y = (0.5*dy)*(f(xcte,yo) + f(xcte,yf) + 2*suma)
        
                    return
                end function
\end{lstlisting}