\section{Programa en Fortran}

\lstset{language=[90]Fortran,
  basicstyle=\ttfamily\scriptsize,
  keywordstyle=\color{red},
  commentstyle=\color{blue},
  breaklines=true,
  showstringspaces=false,
  morecomment=[l]{!\ }% Comment only with space after !
}

\begin{lstlisting}
! Programa elaborado por Jesus Eduardo Loera Casas
! Fecha 01/02/23

    REAL FUNCTION fx(x)
        IMPLICIT NONE
        REAL :: x
        ! Parametros
        REAL :: mo, g, q, u
        g = 9.8; u = 1800; mo = 160000; q = 2500
        ! ecuacion del cohete
        fx = u*log((mo)/(mo-q*x))-g*x
        RETURN 
    END

    REAL FUNCTION romberg_00(a, b)   
        IMPLICIT NONE
        REAL :: a, b, fx
        romberg_00 = 0.5*(b-a)*(fx(a)+fx(b))
        RETURN
    END

    RECURSIVE FUNCTION romberg_n0(n, a, b) result(integral)
        IMPLICIT NONE
        INTEGER :: n, k
        REAL :: a, b, fx, romberg_00, hn, sum, integral
        IF (n==0) THEN
            integral = romberg_00(a, b)
        ELSE 
            hn = (b-a)/(2**n)
            sum = 0
            DO k = 1, 2**(n-1)
                sum = sum + fx(a + (2*k-1)*hn)
            END DO
            integral = 0.5*(romberg_n0(n-1, a, b)) + hn*sum
        END IF
        RETURN
    END

    REAL FUNCTION romberg_nm(n, m, a, b)
        IMPLICIT NONE
        INTEGER :: n, m, i, j
        REAL :: a, b, romberg_ij, aux, romberg_00, romberg_n0, tol
        REAL, DIMENSION ((n+1),(m+1)) :: matrix

        ! Definimos una tolarancia al error con cada iteracion
        tol = 0.00001
            
        matrix(1,1) = romberg_00(a,b)

        DO i = 2, n + 1
            matrix(i,1) = romberg_n0(i-1, a, b)
            ! Evaluamos el criterio de convergencia
            IF (abs(matrix(i,1)-matrix(i-1,1)) .le. tol ) THEN
                write(*,*) "La integral convergio con una" 
                write(*,*) "tolerancia de: ", tol
                romberg_nm = matrix(i,1)
                RETURN
            END IF
        END DO

        DO j = 2, m+1
            DO i = j, n+1
                aux = (4**(j-1))*matrix(i, j-1) - matrix(i-1, j-1)
                romberg_ij = (1.0/((4.0**(j-1))-1))*(aux)
                matrix(i,j) = romberg_ij
                ! Evaluamos el criterio de convergencia
                IF (abs(matrix(i,j)-matrix(i-1,j)) .le. tol ) THEN
                    write(*,*) "La integral convergio."
                    romberg_nm = matrix(i,j)
                    RETURN
                END IF
            END DO
        END DO

        write(*,*)"La integral no convergio con la tolerancia dada."
        romberg_nm = matrix(n+1, m+1)
        RETURN
    END

    PROGRAM main
 
        IMPLICIT NONE
        INTEGER :: n, m
        REAL :: a, b
        REAL :: romberg_nm, int_nm

        n = 10 ; m = 4
        a = 0  ; b = 30

        int_nm = romberg_nm(n, m, a, b)

        WRITE(*,*) "La aproximacion de romber es: ", int_nm

    END PROGRAM main
\end{lstlisting}